In order to evaluate the performance of our design, we will be coding a number 
of well known graph related algorithms in our language and run the proposed interpreter on 
them. The baseline of our evaluation will the the same algorithms implemented in 
C,  compiled by the C compiler and executing the compiled binary.

We are planning to compare both the convenience of representation in our 
language w.r.t C language as well as the runtime performance of the interpreter 
w.r.t the compiled version of the code.

We will be using the number of lines of code as a metric to compare the 
convenience that our language is providing w.r.t C. This comparision can yield 
meaningful results if we are carefull about the following aspects.
\begin{itemize}
  \item Both the code implementations should produce the same output.
  \item In both the code implementation, the backward slice of the output instructions should cover the entire 
  code. In other words, there should not be any redundant code which is not 
  contributing to the generated output. 
  \item Both the code implementation should follow similar coding 
  guidelines~\cite{GNU}.
\end{itemize}

\subsection{Runtime Performance Comparision}

We have implemented a couple of algorithms in both our language and C. The 
algorithms are Transitive Closure of a graph using Floyd Warshall's algorithm, 
           Shortest Path using Dijkstra's algorithm, Minimum Spanning Tree using 
           Prim's algorithm and Chaitin's Optimistic graph coloring algorithm.
The input graph used in all the cases is a randomly generated graph consisting  of vertices = 500, edges = 1000. 
For Graph Coloring, we used the input graph with vertices = 125, edges = 1000. This is 
because it takes a lot of time for our interpreter exection to complete.
In Table~\ref{table:eval_1}, we compared the runtime of our interpreter 
(henceforth called GRI) with 
execution time of compiled binary obtained by compiling the C implementation.

Note that, while implementing the algorithms, we used the basic syntax provided 
by our implementation and \textbf{NOT} used any of the  built-in 
functions like {\tt getTransitiveClosure(), getShortestPath() or getMST()}. 
We call such implementations as fully scripted.

  \begin{table}[]
\centering
\caption{Slowdown of GRI w.r.t C implementation.}
\label{table:eval_1}
\scalebox{.8} {
\begin{tabular}{|l|l|l|l|}
\hline
   \multicolumn{1}{|c|}{Algorithm}                                 & \multicolumn{1}{c|}{\begin{tabular}[c]{@{}c@{}}Fully Scripted\\ time (secs)\end{tabular}} & \multicolumn{1}{c|}{\begin{tabular}[c]{@{}c@{}}C implementation\\ time(secs)\end{tabular}} & Slowdown \\ \hline
Transitive Closure  & 1137.40            & 4.04                   & 281.5    \\ \hline
Shortest Path          & 6.37             & 0.02                     & 318.5     \\ \hline
Minimum Spanning Tree        & 6.34             & 0.02                      & 317.0     \\ \hline
Graph Coloring  &   138.17  & 0.65                              & 212.6         \\ \hline
\end{tabular}}
\end{table}

As we can see that the slowdown are huge. To counter that, we planned to come up 
with following built-in functions.
  \begin{itemize}
    \item \tt{Graph::getAdjMatrix();}
    \item \tt{Graph::getTransitiveClosure();}
    \item \tt{Graph::getShortestPath(string wt, Vertex start, vertex end);}
    \begin{itemize}
      \item Returns the shortest distance from start to each vertex in the 
      shortest path tree.
      \item Returns parent of each vertex in the shortest path tree.
      \item If \tt{end == NULL}, return the shortespath from start to all vertices.
      \item If \tt{end != NULL}, return the shortespath from start to end.
    \end{itemize} 
    \item \tt{Graph::getMST(string wt);}
    \begin{itemize}
      \item Return the parent of each vertex in the minimum snapping tree.
    \end{itemize} 
  \end{itemize}


  Table~\ref{table:eval_2} shows the speedup of our implementation \textbf{with} built-in 
  functions w.r.t  \textbf{without} using built-in functions (i.e. fully 
      Scripted). The graph used in 
  all the cases consist of vertices = 125, edges = 1000.

\begin{table}[]
\centering
\caption{Speedup with built-in functions w.r.t without 
  using built-in functions.}
\label{table:eval_2}
\begin{tabular}{|l|l|l|l|}
\hline
                                    & \multicolumn{1}{c|}{\begin{tabular}[c]{@{}c@{}}With Built-ins\\ time (secs)\end{tabular}} & \multicolumn{1}{c|}{\begin{tabular}[c]{@{}c@{}}Fully Scripted\\ time(secs)\end{tabular}} & Speedup \\ \hline
Transitive Closure & 0.54                                                                                      & 18.19                                                                                    & 33.6    \\ \hline
Shortest Path             & 0.08                                                                                      & 0.51                                                                                     & 6.3     \\ \hline
Minimum Spanning Tree        & 0.05                                                                                      & 0.47                                                                                     & 9.4     \\ \hline
\end{tabular}
\end{table}

Finally, we implemented the above mentioned algorithms using built-in functions 
and compared with C (Figure ~\ref{table:eval_3}). The graph used in all the cases consist of vertices = 500, edges = 1000. For Graph Coloring, we used the graph with vertices = 125, edges = 1000.

  \begin{table}[]
\centering
\caption{Slowdown of our implementation w.r.t C implementation.}
\label{table:eval_3}
\begin{tabular}{|l|l|l|l|}
\hline
                                    & \multicolumn{1}{c|}{\begin{tabular}[c]{@{}c@{}}GRI\\ Time(secs)\end{tabular}} & \multicolumn{1}{c|}{\begin{tabular}[c]{@{}c@{}}C\\ Time(secs)\end{tabular}} & Slowdown \\ \hline
Transitive Closure (Floyd Warshall) & 10.26     & 4.04                               & 2.5     \\ \hline
Shortest Path (Dijkstra)            & 0.09      & 0.02                               & 4.5     \\ \hline
Minimum Spanning Tree (Prim)        & 0.09      & 0.02                               & 4.5     \\ \hline
Graph Coloring (Chaitin Optimistic) &   138.17  & 0.65                              & 212.6         \\ \hline
\end{tabular}
\end{table}

\subsection{Comparision w.r.t Representation Convenience }


