\documentclass[letterpaper]{sig-alternate} \special{papersize=8.5in,11in}
\usepackage{url} \newtheorem{example}{Example}
\newtheorem{definition}{Definition}
%%%%%% Non re-Com packages
\usepackage{pst-node} \usepackage{pst-rel-points}
%%%%%%%%%%%%%%%%%%%%%%%%%%%%%%%%%%%%%%%% %%%%%% New Commands %%%%%%%%%%%%%%%%%%
%%%%%%%%%%%%%%%%%%%%%%%%%%%%%%%%%%%%%%%%

\begin{document}

\title{GRI: Interpreter of a dynamic language for GRaph algorithms}

\numberofauthors{1} 
% reasons) and the remaining two appear in the \additionalauthors section.
%
\author{
%\alignauthor Sandeep Dasgupta\\
\alignauthor Sandeep Dasgupta\\ \affaddr{University Of Illinois at Urbana
  Champaign.}\\ \email{sdasgup3@illinois.edu} } \date{}

\maketitle \begin{abstract} As graphical models are increasingly become popular
in various field, the domain experts often struggle to represent and compute on
such model in a convenient and efficient way. In this project we develop a
dynamically typed language to represent and compute on the graphical models
which provides both the desired convenience without loosing much on the
efficiency.  \end{abstract}

\keywords{Graph Algorithm, dynamically typed language, interpreter} 

%%%%%%%%%%%%%%%%%%%%%%%%%%%%%%%%%%%%%%%%%%%%%%%%%%%%%
\section{Introduction}
%%%%%%%%%%%%%%%%%%%%%%%%%%%%%%%%%%%%%%%%%%%%%%%%%%%%%
As graphical models are increasingly being used in various fields like
biochemistry (genomics), electrical engineering (communication networks and
    coding theory), computer science (algorithms and computation) and
operations research (scheduling), organizational structures, social networking,
           there is a need to represent and allow computation on them in a
convenient and efficient way. This involves (but not limited to)

    \begin{itemize} \item Designing a language which provide an convenient
    interface to the programmer to program those models.  This is essential so
    that even for domain experts who are not coding experts can code and reason
    about their implementation.  Ease of interface could be due to:
    \begin{itemize} \item Expressive power of the language representing those
    models.  \item Intuitive extensibility of the language.  \item Ability of
    the language to provide exploratory programming, where the user may
    experiment with different ideas (without dwelling much into the language
        syntax) before coming to a conclusive one.  \end{itemize} \item
    Designed language need to be efficient in the following sense.
    \begin{itemize} \item Underlying design decisions including data structures
    need to be carefully crafted to achieve expected run-time w.r.t the input
    size.  \item Implementation need to be scalable w.r.t the space/time
    requirements. This is important because most of the graph algorithm
    typically work on huge input sizes.  \end{itemize} \end{itemize}       


%%%%%%%%%%%%%%%%%%%%%%%%%%%%%%%%%%%%%%%%%%%%%%%%%%%%%
\section{Related Work}\label{sec:bgrel}
%%%%%%%%%%%%%%%%%%%%%%%%%%%%%%%%%%%%%%%%%%%%%%%%%%%%%
Our work in mostly inspired by the line of work by GUESS \cite{Adar} and
Graphal \cite{Graphal}.
GUESS, a novel system for graph
exploration that combines an interpreted language with a
graphical front end that allows researchers to rapidly prototype
and deploy new visualizations. GUESS also contains a novel,
interactive interpreter that connects the language and interface in
a way that facilities exploratory visualization tasks. Our
language, Gython, is a domain-specific embedded language
which provides all the advantages of Python with new, graph
specific operators, primitives, and shortcuts.

Graphal is an interpreter of a programming language that is mainly oriented to
graph algorithms. There is a command line interpreter and a graphical
integrated development environment. The IDE contains text editor for
programmers, compilation and script output, advanced debugger and visualization
window. The progress of the interpreted and debugged graph algorithm can be
displayed in 3D scene.

Our language design is very similar to above two work. But we additionally provided
built-in functions for basic graph algorithms. This not only help us getting
convinient short hand notations to compute those basic algorithms, but also we gain on 
performance due the fact that those basic algorithms are now compiled.



%%%%%%%%%%%%%%%%%%%%%%%%%%%%%%%%%%%%%%%%%%%%%%%%%%%%%
\section{A Motivating Example}\label{sec:motiv}
%%%%%%%%%%%%%%%%%%%%%%%%%%%%%%%%%%%%%%%%%%%%%%%%%%%%%
In this section we will provide some insight of designed language
using a motivating example.
Lets consider the graph 




%%%%%%%%%%%%%%%%%%%%%%%%%%%%%%%%%%%%%%%%%%%%%%%%%%%%%
\section{Definitions and Notations} \label{sec:Formal_Definitions}
%%%%%%%%%%%%%%%%%%%%%%%%%%%%%%%%%%%%%%%%%%%%%%%%%%%%%


%%%%%%%%%%%%%%%%%%%%%%%%%%%%%%%%%%%%%%%%%%%%%%%%%%%%%
\subsection{Interprocedural Analysis} \label{Interprocedural_Analysis}
%%%%%%%%%%%%%%%%%%%%%%%%%%%%%%%%%%%%%%%%%%%%%%%%%%%%%



\bibliographystyle{abbrv}
%\bibliography{sigproc}  

\end{document}
