\documentclass[letterpaper]{sig-alternate} \special{papersize=8.5in,11in}
\usepackage{url} 
\usepackage{graphics}
\usepackage{epsfig}
\usepackage{algorithm}
\usepackage{verbatim}
\usepackage{listings}
\usepackage{hyperref}
\usepackage{pst-coil}
\usepackage{color}

\newtheorem{example}{Example}
\newtheorem{definition}{Definition}
\newcommand{\fig}[1]{\includegraphics[scale=.5]{#1}}
\newcommand{\cmt}[1]{}

%%%%%% Non re-Com packages
\usepackage{pst-node} \usepackage{pst-rel-points}
%%%%%%%%%%%%%%%%%%%%%%%%%%%%%%%%%%%%%%%% %%%%%% New Commands %%%%%%%%%%%%%%%%%%
%%%%%%%%%%%%%%%%%%%%%%%%%%%%%%%%%%%%%%%%
\definecolor{codegreen}{rgb}{0,0.6,0}
\definecolor{codegray}{rgb}{0.5,0.5,0.5}
\definecolor{codepurple}{rgb}{0.58,0,0.82}
\definecolor{backcolour}{rgb}{0.95,0.95,0.92}
\lstdefinestyle{mystyle}{
    %backgroundcolor=\color{backcolour},   
    commentstyle=\color{codegreen},
    keywordstyle=\color{magenta},
    %numberstyle=\tiny\color{codegray},
    stringstyle=\color{codepurple},
    basicstyle=\footnotesize,
    breakatwhitespace=false,         
    breaklines=true,                 
    captionpos=b,                    
    keepspaces=true,                 
    %numbers=left,                    
    %numbersep=5pt,                  
    showspaces=false,                
    showstringspaces=false,
    showtabs=false,                  
    tabsize=2
}
 
\lstset{style=mystyle,frameround=fttt}




\begin{document}

\title{GRI: Interpreter of a dynamic language for GRaph algorithms}

\numberofauthors{1} 
% reasons) and the remaining two appear in the \additionalauthors section.
%
\author{
%\alignauthor Sandeep Dasgupta\\
\alignauthor Sandeep Dasgupta\\ \affaddr{University Of Illinois at Urbana
  Champaign.}\\ \email{sdasgup3@illinois.edu} } \date{}

\maketitle \begin{abstract} As graphical models are increasingly become popular
in various field, the domain experts often struggle to represent and compute on
such model in a convenient and efficient way. In this project we develop a
dynamically typed language to represent and compute on the graphical models
which provides both the desired convenience without loosing much on the
efficiency.  \end{abstract}

\keywords{Graph Algorithm, dynamically typed language, interpreter} 

%%%%%%%%%%%%%%%%%%%%%%%%%%%%%%%%%%%%%%%%%%%%%%%%%%%%%
\section{Introduction}
%%%%%%%%%%%%%%%%%%%%%%%%%%%%%%%%%%%%%%%%%%%%%%%%%%%%%
As graphical models are increasingly being used in various fields like
biochemistry (genomics), electrical engineering (communication networks and
    coding theory), computer science (algorithms and computation) and
operations research (scheduling), organizational structures, social networking,
           there is a need to represent and allow computation on them in a
convenient and efficient way. This involves (but not limited to)

    \begin{itemize} \item Designing a language which provide an convenient
    interface to the programmer to program those models.  This is essential so
    that even for domain experts who are not coding experts can code and reason
    about their implementation.  Ease of interface could be due to:
    \begin{itemize} \item Expressive power of the language representing those
    models.  \item Intuitive extensibility of the language.  \item Ability of
    the language to provide exploratory programming, where the user may
    experiment with different ideas (without dwelling much into the language
        syntax) before coming to a conclusive one.  \end{itemize} \item
    Designed language need to be efficient in the following sense.
    \begin{itemize} \item Underlying design decisions including data structures
    need to be carefully crafted to achieve expected run-time w.r.t the input
    size.  \item Implementation need to be scalable w.r.t the space/time
    requirements. This is important because most of the graph algorithm
    typically work on huge input sizes.  \end{itemize} \end{itemize}       


%%%%%%%%%%%%%%%%%%%%%%%%%%%%%%%%%%%%%%%%%%%%%%%%%%%%%
\section{Related Work}\label{sec:bgrel}
%%%%%%%%%%%%%%%%%%%%%%%%%%%%%%%%%%%%%%%%%%%%%%%%%%%%%
Our work in mostly inspired by the line of work by GUESS \cite{Adar} and
Graphal \cite{Graphal}.
GUESS, a novel system for graph
exploration that combines an interpreted language with a
graphical front end that allows researchers to rapidly prototype
and deploy new visualizations. GUESS also contains a novel,
interactive interpreter that connects the language and interface in
a way that facilities exploratory visualization tasks. Our
language, Gython, is a domain-specific embedded language
which provides all the advantages of Python with new, graph
specific operators, primitives, and shortcuts.

Graphal is an interpreter of a programming language that is mainly oriented to
graph algorithms. There is a command line interpreter and a graphical
integrated development environment. The IDE contains text editor for
programmers, compilation and script output, advanced debugger and visualization
window. The progress of the interpreted and debugged graph algorithm can be
displayed in 3D scene.

Our language design is very similar to above two work. But we additionally provided
built-in functions for basic graph algorithms. This not only help us getting
convinient short hand notations to compute those basic algorithms, but also we gain on 
performance due the fact that those basic algorithms are now compiled.



%%%%%%%%%%%%%%%%%%%%%%%%%%%%%%%%%%%%%%%%%%%%%%%%%%%%%
\section{A Motivating Example}\label{sec:motiv}
%%%%%%%%%%%%%%%%%%%%%%%%%%%%%%%%%%%%%%%%%%%%%%%%%%%%%
In this section we will provide some insight of designed language
using some motivating examples.

\begin{figure*}
\begin{center}
\fig{Figs/1}
\end{center}
\end{figure*}

\begin{figure}
\begin{center}
\fig{Figs/2}
\end{center}
\end{figure}

\begin{figure}
\begin{center}
      {\small \tt
        \begin{tabular}[b]{rl}
          &function main(argv) main() \{ \\
	  {\em \scriptsize S1:}& \quad g = graph(); \\
	  {\em \scriptsize S2:}& \quad \ldots        \\
          {\em \scriptsize S3:}& \quad g.setDirected(true); \\
          {\em \scriptsize S4:}& \quad \ldots \\ 
          {\em \scriptsize S5:}& \quad v0  = g.createVertex(); \\
          {\em \scriptsize S6:}& \quad v0.\_\_id     = 0; \\
          {\em \scriptsize S7:}& \quad v0.\_\_DPT\_x  = 1; \\
          {\em \scriptsize S8:}& \quad v0.\_\_DPT\_y  = 0; \\
          {\em \scriptsize S9:}& \quad v0.\_\_MGR    = 1; \\
          {\em \scriptsize S10:}& \quad \ldots \\ 
          {\em \scriptsize S11:}& \quad v3  = g.createVertex(); \\
          {\em \scriptsize S12:}& \quad v3.\_\_id     = 3; \\
          {\em \scriptsize S13:}& \quad v3.\_\_DPT\_x  = 1; \\
          {\em \scriptsize S14:}& \quad v3.\_\_DPT\_y  = 0; \\
          {\em \scriptsize S15:}& \quad v3.\_\_MGR    = 0; \\
          {\em \scriptsize S16:}& \quad \ldots \\ 
          {\em \scriptsize S17:}& \quad g.createEdge(v0,v3); \\
	  &\}
        \end{tabular}
      }
\end{center}
\end{figure}

\begin{figure}
\begin{center}
      {\small \tt
        \begin{tabular}[b]{rl}
          &function main(argv) main() \{ \\
          {\em \scriptsize S1:}&  \quad  g = graph();\\
          {\em \scriptsize S2:}&  \quad  g.loadFromFile(argv[0]);\\
          {\em \scriptsize S3:}&  \quad  displayAdjMatrix(g.getAdjacencyMatrix(), g.getVertices()); \\
          {\em \scriptsize S4:}&  \quad  \ldots \\
          {\em \scriptsize S5:}&  \quad  dpt\_x\_employee  = g.getVertexSetWithProperty("\_\_DPT\_x", 1.0); \\
          {\em \scriptsize S6:}&  \quad  mgr\_employee    = g.getVertexSetWithProperty("\_\_MGR", 1.0); \\
          {\em \scriptsize S7:}&  \quad  \ldots \\
          {\em \scriptsize S8:}&  \quad /* Set of \_\_DPT\_x employees who are \_\_MGR as well */  \\
          {\em \scriptsize S9:}&  \quad dpt\_x\_AND\_mgr = dpt\_x\_employee.intersection(mgr\_employee); \\
          {\em \scriptsize S10:}&  \quad  \ldots \\
          {\em \scriptsize S11:}&  \quad println("Set of \_\_DPT\_x employees who are \_\_MGR as well");  \\
          {\em \scriptsize S12:}&  \quad foreach(employee; dpt\_x\_AND\_mgr) \{ \\
          {\em \scriptsize S13:}&  \quad \quad println(" " + employee.\_\_id + " "); \\
          {\em \scriptsize S14:}&  \quad \}   \\
          {\em \scriptsize S15:}&  \quad  \ldots \\
          {\em \scriptsize S16:}&  \quad /* Set of \_\_DPT\_x employees who are NOT  \_\_MGR */ \\
          {\em \scriptsize S17:}&  \quad dpt\_x\_MINUS\_mgr = dpt\_x\_employee.difference(mgr\_employee); \\
          {\em \scriptsize S18:}&  \quad    \ldots \\
          {\em \scriptsize S19:}&  \quad it = dpt\_x\_MINUS\_mgr.iterator(); \\
          {\em \scriptsize S20:}&  \quad    \ldots \\
          {\em \scriptsize S21:}&  \quad println("Set of \_\_DPT\_x employees who are NOT  \_\_MGR"); \\
          {\em \scriptsize S22:}&  \quad while(it.hasNext()) \{\\
          {\em \scriptsize S23:}&  \quad \quad employee = it.next();\\
          {\em \scriptsize S24:}&  \quad \quad println(" " + employee.\_\_id + " ");\\
          {\em \scriptsize S25:}&  \quad \}  \\
          {\em \scriptsize S26:}&  \quad  \ldots \\
          {\em \scriptsize S27:}&  \quad // Email Exchanges from MGRs to non-MGR DPT\_x employee \\
          {\em \scriptsize S28:}&  \quad emailExchanges = mgr\_employee -> dpt\_x\_MINUS\_mgr;\\
          {\em \scriptsize S29:}&  \quad  \ldots \\
          {\em \scriptsize S30:}&  \quad // Email Exchanges from non-MGR DPT\_x employee to MGRs*/\\
          {\em \scriptsize S31:}&  \quad emailExchanges = mgr\_employee <- dpt\_x\_MINUS\_mgr;\\
          {\em \scriptsize S32:}&  \quad  \ldots \\
          {\em \scriptsize S33:}&  \quad // Email Exchanges between non-MGR DPT\_x employee and MGRs*/\\
          {\em \scriptsize S34:}&  \quad emailExchanges = mgr\_employee <-> dpt\_x\_MINUS\_mgr;\\
          {\em \scriptsize S35:}&  \quad  \ldots \\
	  &\}
        \end{tabular}
      }
\end{center}
\end{figure}


\begin{example}{\rm
In the following figure we have a directed graph where the nodes represent the employees
and the edges between them represents the mail conversation from one employee to 
other. For example, {\tt Node 0} represemts an employye in {\tt DPT\_x} who is a 
Manager as well (this is attributed by the {\tt MGR} tag). Similarly, {\tt Node 
3} represents a {\tt DPT\_x} employee. The edge between {\tt Node 3} and {\tt 
Node 0} represents the email conversation from employee {\tt Node 3} to 
  employee node {\tt Node 0}.

  Now lets first talk about the ways to represent this graph. One way is as 
  shown in fig. At line {\tt S1}, a new graph variable is created. {\tt S3} sets that
  the graph is directed. {\tt S5} create a node {\tt Node 0} of this graph. As nodes 
  and edges are the basic building block of a graph we have made them the frst 
  class objects which allows users to access them directly. Lines {\tt S6} - 
  {\tt S9} sets various properties of the vertex. For example, it says that the 
vertex has {\tt \_\_id = 0 }, it belongs to  {\tt \_\_DPT\_x}, does not belongs 
  to {\tt \_\_DPT\_y} and its {\tt \_\_MGR} property is true. Similar properties 
  are set for vertex node with {\tt \_\_id = 3}.
  And finally a directed  edge between them is created at line {\tt S17}. 

  One thing to note here is that the methods 
  like {\tt graph}, {\tt setDirected}, {\tt createVertex} and {\tt createEdge} 
are all built-in compiled functions. 

  Figure shows another way to represent graph. It uses an input file to be fed to
  the interpreted program. The format of the input file is shon in Figure.
  Line {\tt S2} loads the input file using command line arguments. At {\\t S3}, we can 
  see two built-in functions {\tt getAdjacencyMatrix} \& {\tt getVertices} on graph
  which respectively gives an array of array representing the adjacency matrix representation of the 
  graph and a set of vertices in the graph. {\tt displayAdjMatrix} is just a method call, the definition of 
  which is not shown for brevity. Now lets try to solve certain queries from the graph at Fogure.
  In case we want to get all the nodes in the graph who are {\tt \_\_DPT\_x} employees, then
  the query to get all those nodes is at line {\tt S5}. Similarly, the query at line {\tt S6}, gives
  the nodes for wehich the property {\tt \_\_MGR} is set to true. 
  One thing to note here that the return value of both the queries are sets which are amenable tos et operations.

  For example, in cse we want to get all the employees who are both depratment {\tt \_\_DPT\_x} empoyee and managers
  , we can get that using the set intersection as shown at line {\tt S9}. Line {\tt S8}, shows the multilien 
  comment we support. Line {\tt S9} shows a way to iterate over the set elements using foreach construct.

  Again, if we want to get the  set of {\t \_\_DPT\_x} employees who are \textbf{NOT} {\tt \_\_MGR},
  we can write a query like at line {\tt S17}. Line {\tt S19} shows another ways to get an iteration to itearte on 
    composite data structures. 

 Now once we have two sets of nodes each with a specific property, we can query for the edges between them.
 For example, lines {\tt S28}, {\tt 31} and {\tt 34} gives the set of edges emanating from one set of nodes
 to onother set of nodes. Note that operators $\leftarrow$ and $\rightarrow$ are applicable to undirected graphs
 and the operator $\leftrightarrow$ is applicable to both directed and undirected graphs. For directed graphs,
 it gives all the bidirectional edges between two node sets and for undirected graphs, it returns all the edges between
   two node sets.
    \hfill\psframebox{}}
\end{example}

\begin{example} {\rm

  Let consider another example implementation
    the depth first traversal on a graph.

\begin{lstlisting}[language=C++,basicstyle=\tiny]
  define("NUM_VERTICES", "10");
  define("NOTVISITED", "0");
  define("VISITED", "1");

  function dfs(v, dfsorder)
  {
      if(v.visit == VISITED)
        return;

      println("vertex visited: " + v.num);
      v.visit = VISITED;

      dfsorder.pushBack(v);

      foreach(neighbor; v.getNeighbors())
        dfs(neighbor, dfsorder);
  }

  function main(argv)
  {
    g = graph();
    g.loadFromFile(argv[0]);

    dfsorder = array(0);
    dfs(first, dfsorder);

    for(vertex: dfsorder ) {
      println(" " + vertex.__id + " " );
    }
  }

\end{lstlisting}

    \hfill\psframebox{}}
\end{example}



%%%%%%%%%%%%%%%%%%%%%%%%%%%%%%%%%%%%%%%%%%%%%%%%%%%%%
\section{Definitions and Notations} \label{sec:Formal_Definitions}
%%%%%%%%%%%%%%%%%%%%%%%%%%%%%%%%%%%%%%%%%%%%%%%%%%%%%


%%%%%%%%%%%%%%%%%%%%%%%%%%%%%%%%%%%%%%%%%%%%%%%%%%%%%
\subsection{Interprocedural Analysis} \label{Interprocedural_Analysis}
%%%%%%%%%%%%%%%%%%%%%%%%%%%%%%%%%%%%%%%%%%%%%%%%%%%%%



\bibliographystyle{abbrv}
%\bibliography{sigproc}  

\end{document}

\cmt{
}
